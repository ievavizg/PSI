\documentclass{VUMIFPSkursinis}
\usepackage{algorithmicx}
\usepackage{algorithm}
\usepackage{algpseudocode}
\usepackage{amsfonts}
\usepackage{amsmath}
\usepackage{bm}
\usepackage{caption}
\usepackage{color}
\usepackage{float}
\usepackage{graphicx}
\usepackage{listings}
\usepackage{subfig}
\usepackage{wrapfig}
\usepackage{sectsty}

\usepackage{enumitem}
%PAKEISTA, tarpai tarp sąrašo elementų
\setitemize{noitemsep,topsep=0pt,parsep=0pt,partopsep=0pt}
\setenumerate{noitemsep,topsep=0pt,parsep=0pt,partopsep=0pt}
\allsectionsfont{\centering}
% Titulinio aprašas
\university{Vilniaus universitetas}
\faculty{Matematikos ir informatikos fakultetas}
\department{Programų sistemų katedra}
\papertype{Programų sistemų inžinerijos I laboratorinis darbas}
\title{Internetinis aukcionas}
\titleineng{Digital Auction}
\status{2 kurso 4 grupės studentai}
\author{Andrejus Voitovas}
\secondauthor{Eglė Puodžiūnaitė}
\thirdauthor{Kasparas Kralikas}
\fourthauthor{Ieva Vizgirdaitė} % Pridėti antrą autorių
\supervisor{asist. dr. Vytautas Valaitis}
\date{Vilnius – \the\year}

% Nustatymai
% \setmainfont{Palemonas}   % Pakeisti teksto šriftą į Palemonas (turi būti įdiegtas sistemoje)
\bibliography{bibliografija}

\begin{document}
% PAKEISTA
\maketitle
\cleardoublepage\pagenumbering{arabic}
\setcounter{page}{2}
\sectionnonum{ANOTACIJA}
Anotacija?
\newpage
%TURINYS
\tableofcontents

\sectionnonum{ĮVADAS}
Ivadas?
\newpage
\section{REIKALAVIMAI}
Andrius, Egle
\newpage
\section{STRUKTŪRINIS DALYKINĖS SRITIES MODELIS}
Kasparas
\newpage
\section{UŽDUOTYS}
Šioje dokumento dalyje yra pateikiamos sistemos atliekamos užduotys  (\ref{fig:usecase} pav.) Pateikiamas pagrindinis scenarijus ir alternatyvūs.\\
\begin{figure}[H]
\centering
\includegraphics[width=\linewidth]{img/UseCaseDiagram.png}
\label{fig:usecase}
\caption{Sistemoje atliekamos užduotys}
\end{figure}

\subsection{Naudotojo atliekamos užduotys}
Šiame skyrelyje išskiriamos visos sistemos naudotojo užduotys (\ref{fig:usercd} pav.), jas aprašant pateikiamas ne tik pagrindinis scenarijus, bet ir alternatyvūs. Taip pat atliekama robastiškumo analizė kiekvienai užduočiai.

\begin{figure}[H]
\centering
\includegraphics[width=\linewidth]{img/userusecased.png}
\label{fig:usercd}
\caption{Naudotojo atliekamos užduotys}
\end{figure}

\subsubsection{Užduotis - Registruotis}
Skyriuje pateikiamas užduoties registruotis aprašymas. Pateikiama užduoties robastiškumo analizės diagrama.\\
\textbf{Užduotis:}  Registruotis \\
\textbf{Scenarijus:} Naudotojas prisijungimo lange paspaudžia mygtuką registruotis. Sistema atidaro registracijos langą. Naudotojas suveda prisijungimo vardą, slaptažodį  \\
\textbf{Alternatyvūs scenarijai:} \\
\textbf{Nuoroda į reikalavimą: } 


\newpage

\sectionnonum{REZULTATAI}
\end{document}
