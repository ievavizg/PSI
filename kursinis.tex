\documentclass{VUMIFPSkursinis}
\usepackage{algorithmicx}
\usepackage{algorithm}
\usepackage{algpseudocode}
\usepackage{amsfonts}
\usepackage{amsmath}
\usepackage{bm}
\usepackage{caption}
\usepackage{color}
\usepackage{float}
\usepackage{graphicx}
\usepackage{listings}
\usepackage{subfig}
\usepackage{wrapfig}
\usepackage{sectsty}
\usepackage{enumerate}
\usepackage{longtable}
\usepackage[export]{adjustbox}
\usepackage{rotating}
\usepackage{hhline}
\usepackage{multirow}
\usepackage{tabularx}
\usepackage{array}
\usepackage{booktabs,calc}
\usepackage{tabularx,colortbl}
\usepackage[table]{xcolor}  
\usepackage{longtable}  

\usepackage{enumitem}
%PAKEISTA, tarpai tarp sąrašo elementų
\setitemize{noitemsep,topsep=0pt,parsep=0pt,partopsep=0pt}
\setenumerate{noitemsep,topsep=0pt,parsep=0pt,partopsep=0pt}
\allsectionsfont{\centering}
% Titulinio aprašas
\university{Vilniaus universitetas}
\faculty{Matematikos ir informatikos fakultetas}
\department{Programų sistemų katedra}
\papertype{Programų sistemų inžinerijos I laboratorinis darbas}
\title{Internetinis aukcionas}
\titleineng{Digital Auction}
\status{2 kurso 4 grupės studentai}
\author{Andrejus Voitovas}
\secondauthor{Eglė Puodžiūnaitė}
\thirdauthor{Kasparas Kralikas}
\fourthauthor{Ieva Vizgirdaitė} % Pridėti antrą autorių
\supervisor{asist. dr. Vytautas Valaitis}
\date{Vilnius – \the\year}

% Nustatymai
% \setmainfont{Palemonas}   % Pakeisti teksto šriftą į Palemonas (turi būti įdiegtas sistemoje)
\bibliography{bibliografija}

\begin{document}
% PAKEISTA
\maketitle
\cleardoublepage\pagenumbering{arabic}
\setcounter{page}{2}
\sectionnonum{ANOTACIJA}
Anotacija?
\newpage
%TURINYS
\tableofcontents

\sectionnonum{ĮVADAS}
Ivadas?
\newpage
\section{FUNKCINIAI REIKALAVIMAI}
Šiame skyriuje pateikiami funkciniai reikalavimai – nagrinėjami scenarijai, ką sistema turi daryti, kaip elgtis vienu ar kitu atveju.

\subsection{Internetinės svetainės langai}
\begin{table}[H]
	\caption{Funkciniai reikalavimai. Internetinės svetainės langai.}
	\begin{tabular}{|p{1cm}|p{1cm}|p{11,5cm}|p{3,5cm}|}
		\hline 
		\rowcolor{gray!50}
		\multicolumn{2}{|c|}{{\bfseries Kodas}}&
		\multicolumn{1}{c|}{{\bfseries Reikalavimas}}&
		\multicolumn{1}{c|}{{\bfseries Svarba}}\\
		\hline
		\rowcolor{lightgray}
		\multicolumn{4}{|c|}{Internetinės svetainės langai}\\		
		
		\hline
		\multicolumn{2}{|c|}{FR 1.1}&
		{Svetainės langai „Registracija“, „Prisijungimas“, „Pagrindinis puslapis“, „Taisyklės“ turi būti matomi visiems net ir neprisijungusiems naudotojams.
		}&		
		\multicolumn{1}{c|}{Būtina}\\
		\hline
		\multicolumn{1}{|c}{}&
		\multicolumn{1}{c|}{FR 1.2}&
		{Svetainės langai „Mano profilis“, „Prekės įkėlimas“, „Prekės aukcionas“ turi būti matomi visiems prisijungusiems naudotojams.
		}&		
		\multicolumn{1}{c|}{Būtina}\\
		\hline
		\multicolumn{1}{|c}{}&
		\multicolumn{1}{c|}{FR 1.3}&
		{Svetainės langai „Visi aukcionai“, „Naudotojai“  turi būti matomas sistemos adminstratoriams.
		}&
		\multicolumn{1}{c|}{Būtina}\\	
		\hline		
	\end{tabular}		
\end{table}

\subsection{Registracija}
\begin{table}[H]
	\caption{Funkciniai reikalavimai. Registracija.}
	\begin{tabular}{|p{1cm}|p{1cm}|p{11,5cm}|p{3,5cm}|}
		\hline 
		\rowcolor{gray!50}
		\multicolumn{2}{|c|}{{\bfseries Kodas}}&
		\multicolumn{1}{c|}{{\bfseries Reikalavimas}}&
		\multicolumn{1}{c|}{{\bfseries Svarba}}\\
		\hline
		\rowcolor{lightgray}
		\multicolumn{4}{|c|}{Registracija}\\		
		
		\hline
		\multicolumn{2}{|c|}{FR 2.1}&
		{Naudotojui suvedus visą reikiamą informaciją ir nuspaudus mygtuką registruotis, jis turi būti priregistruojamas internetinėje sveitainėje.
		}&		
		\multicolumn{1}{c|}{Būtina}\\
		\hline
		\multicolumn{1}{|c}{}&
		\multicolumn{1}{c|}{FR 2.2}&
		{Naudotojui nesuvedus informacijos į laukus pažymėtus žvaigždute jis neturi būti priregistruojamas nei registracijos puslapyje turi būti išmetamas klaidos pranešimas, pranešantis, jog reikia užpildyti visus privalomus laukus.
		}&		
		\multicolumn{1}{c|}{Būtina}\\
		\hline	
		\multicolumn{1}{|c}{}&
		\multicolumn{1}{c|}{FR 2.3}&
		{Registruojantis naudotojas turi įvesti internetinėje svetainėje dar nepriregistruotą el. pašto adresą.
		}&
		\multicolumn{1}{c|}{Būtina}\\									
		\hline
	\end{tabular}		
\end{table}

\subsection{Prisijungimas}
\begin{table}[H]
	\caption{Funkciniai reikalavimai. Prisijungimas.}
	\begin{tabular}{|p{1cm}|p{1cm}|p{11,5cm}|p{3,5cm}|}
		\hline 
		\rowcolor{gray!50}
		\multicolumn{2}{|c|}{{\bfseries Kodas}}&
		\multicolumn{1}{c|}{{\bfseries Reikalavimas}}&
		\multicolumn{1}{c|}{{\bfseries Svarba}}\\
		\hline
		\rowcolor{lightgray}
		\multicolumn{4}{|c|}{Prisijungimas}\\		
		
		\hline
		\multicolumn{2}{|c|}{FR 3.1}&
		{Naudotojui suvedus tinkamus prisijungimo duomenis jis turi būti prijungiamas prie sistemos.
		}&		
		\multicolumn{1}{c|}{Būtina}\\
		\hline
		\multicolumn{1}{|c}{}&
		\multicolumn{1}{c|}{FR 3.2}&
		{Naudotojui netinkamai įvedus prisijungimo duomenis jis neturi būti prijungiamas prie sistemos ir turi būti išmetamas klaidos pranešimas.
		}&		
		\multicolumn{1}{c|}{Būtina}\\
		\hline	
		\multicolumn{1}{|c}{}&
		\multicolumn{1}{c|}{FR 3.3}&
		{Naudotojų bandymų prisijungti prie sistemos skaičius neturi būti ribojamas.
		}&
		\multicolumn{1}{c|}{Būtina}\\									
		\hline
	\end{tabular}		
\end{table}

\subsection{Atsijungimas}

\begin{table}[H]
	\caption{Funkciniai reikalavimai. Atsijungimas.}
	\begin{tabular}{|p{1cm}|p{1cm}|p{11,5cm}|p{3,5cm}|}
		\hline 
		\rowcolor{gray!50}
		\multicolumn{2}{|c|}{{\bfseries Kodas}}&
		\multicolumn{1}{c|}{{\bfseries Reikalavimas}}&
		\multicolumn{1}{c|}{{\bfseries Svarba}}\\
		\hline
		\rowcolor{lightgray}
		\multicolumn{4}{|c|}{Atsijungimas}\\		
		
		\hline
		\multicolumn{2}{|c|}{FR 4.1}&
		{Naudotojas paspaudęs mygtuką atsijungti turi būti atjungiamas nuo sistemos.
		}&		
		\multicolumn{1}{c|}{Būtina}\\
		\hline
	\end{tabular}		
\end{table}

\subsection{Paskyros valdymas}

\begin{table}[H]
	\caption{Funkciniai reikalavimai. Paskyros valdymas.}
	\begin{tabular}{|p{1cm}|p{1cm}|p{11,5cm}|p{3,5cm}|}
		\hline 
		\rowcolor{gray!50}
		\multicolumn{2}{|c|}{{\bfseries Kodas}}&
		\multicolumn{1}{c|}{{\bfseries Reikalavimas}}&
		\multicolumn{1}{c|}{{\bfseries Svarba}}\\
		\hline
		\rowcolor{lightgray}
		\multicolumn{4}{|c|}{Paskyros valdymas}\\				
		\hline
		\multicolumn{2}{|c|}{FR 5.1}&
		{Naudotojui paspaudus mygtuką „Mano profilis“ turi būti matoma visa žinoma informacija apie naudotoją.
		}&		
		\multicolumn{1}{c|}{Būtina}\\
		\hline
		\multicolumn{1}{|c}{}&
		\multicolumn{1}{c|}{FR 5.2}&
		{Naudotojui paspaudus mygtuką „Pakeisti slaptažodį“, įvedus tinkamą seną ir naują slaptažodžius bei paspaudus mygtuką „Patvirtinti“ slaptažodis turi būti pakeičiamas į naują.
		}&		
		\multicolumn{1}{c|}{Būtina}\\
		\hline
		\multicolumn{1}{|c}{}&
		\multicolumn{1}{c|}{FR 5.3}&
		{Naudotojui paspaudus mygtuką „Pakeisti slaptažodį“ ir įvedus netinkamą seną slaptažodį turi būti išvedamas klaidos pranešimas.
		}&
		\multicolumn{1}{c|}{Būtina}\\	
		\hline		
	\end{tabular}		
\end{table}

\subsection{Taisyklės}
\begin{table}[H]
	\caption{Funkciniai reikalavimai. Taisyklės}
	\begin{tabular}{|p{1cm}|p{1cm}|p{11,5cm}|p{3,5cm}|}
		\hline 
		\rowcolor{gray!50}
		\multicolumn{2}{|c|}{{\bfseries Kodas}}&
		\multicolumn{1}{c|}{{\bfseries Reikalavimas}}&
		\multicolumn{1}{c|}{{\bfseries Svarba}}\\
		\hline
		\rowcolor{lightgray}
		\multicolumn{4}{|c|}{Taisyklės}\\		
		
		\hline
		\multicolumn{2}{|c|}{FR 6.1}&
		{Taisyklės turi būti matomos puslapyje „Taisyklės“.
		}&		
		\multicolumn{1}{c|}{Būtina}\\
		\hline
		\multicolumn{1}{|c}{}&
		\multicolumn{1}{c|}{FR 6.2}&
		{Visas taisyklių sąrašas turi būti pateikiamas viename puslapyje.
		}&		
		\multicolumn{1}{c|}{Būtina}\\
		\hline	
		\multicolumn{1}{|c}{}&
		\multicolumn{1}{c|}{FR 6.3}&
		{Taisyklės turi būti matomos visiems prisijungusiems ir neprisijungusiems naudotojams.
		}&
		\multicolumn{1}{c|}{Būtina}\\									
		\hline
	\end{tabular}		
\end{table}

\subsection{Pagrindinis puslapis}
\begin{table}[H]
	\caption{Funkciniai reikalavimai. Pagrindinis puslapis}
	\begin{tabular}{|p{1cm}|p{1cm}|p{11,5cm}|p{3,5cm}|}
		\hline 
		\rowcolor{gray!50}
		\multicolumn{2}{|c|}{{\bfseries Kodas}}&
		\multicolumn{1}{c|}{{\bfseries Reikalavimas}}&
		\multicolumn{1}{c|}{{\bfseries Svarba}}\\
		\hline
		\rowcolor{lightgray}
		\multicolumn{4}{|c|}{Pagrindinis puslapis}\\		
		
		\hline
		\multicolumn{2}{|c|}{FR 7.1}&
		{Aktyvūs aukcionai turi būti matomi puslapyje „Pagrindinis puslapis“.
		}&		
		\multicolumn{1}{c|}{Būtina}\\
		\hline
		\multicolumn{1}{|c}{}&
		\multicolumn{1}{c|}{FR 7.2}&
		{Visas aktyvių aukcionų sąrašas turi būti pateikiamas viename puslapyje.
		}&		
		\multicolumn{1}{c|}{Būtina}\\
		\hline	
		\multicolumn{1}{|c}{}&
		\multicolumn{1}{c|}{FR 7.3}&
		{Pagrindinis puslapis turi būti matomos visiems prisijungusiems ir neprisijungusiems naudotojams.
		}&
		\multicolumn{1}{c|}{Būtina}\\									
		\hline
		\multicolumn{1}{|c}{}&
		\multicolumn{1}{c|}{FR 7.4}&
		{Prisijungusiam naudotojui paspaudus ant aktyvaus aukciono mygtuko „Daugiau informacijos“ turi būti atidaromas puslapis „Prekės aukcionas“.
		}&
		\multicolumn{1}{c|}{Būtina}\\									
		\hline
	\end{tabular}		
\end{table}

\subsection{Prekės įkėlimas}
\begin{table}[H]
	\caption{Funkciniai reikalavimai. Prekės įkėlimas}
	\begin{tabular}{|p{1cm}|p{1cm}|p{11,5cm}|p{3,5cm}|}
		\hline 
		\rowcolor{gray!50}
		\multicolumn{2}{|c|}{{\bfseries Kodas}}&
		\multicolumn{1}{c|}{{\bfseries Reikalavimas}}&
		\multicolumn{1}{c|}{{\bfseries Svarba}}\\
		\hline
		\rowcolor{lightgray}
		\multicolumn{4}{|c|}{Prekės įkėlimas}\\		
		
		\hline
		\multicolumn{2}{|c|}{FR 8.1}&
		{Puslapis „Prekės įkėlimas“ turi būti matomas visiems prijungusiems naudotojams.
		}&		
		\multicolumn{1}{c|}{Būtina}\\
		\hline
		\multicolumn{1}{|c}{}&
		\multicolumn{1}{c|}{FR 8.2}&
		{Naudotojui suvedus visą reikiamą informaciją apie aukciono prekę bei paspaudus mygtuką „Patvirtinti“ turi būti sukuriamas aukcionas.
		}&		
		\multicolumn{1}{c|}{Būtina}\\
		\hline	
		\multicolumn{1}{|c}{}&
		\multicolumn{1}{c|}{FR 8.3}&
		{Naudotojui nesuvedus informacijos į laukus pažymėtus žvaigždute aukcionas neturi būti sukuriamas nei prekės įkėlimo puslapyje turi būti išmetamas klaidos pranešimas, pranešantis, jog reikia užpildyti visus privalomus laukus.
		}&
		\multicolumn{1}{c|}{Būtina}\\									
		\hline
	\end{tabular}		
\end{table}

\subsection{Prekės aukcionas}
\begin{table}[H]
	\caption{Funkciniai reikalavimai. Prekės aukcionas}
	\begin{tabular}{|p{1cm}|p{1cm}|p{11,5cm}|p{3,5cm}|}
		\hline 
		\rowcolor{gray!50}
		\multicolumn{2}{|c|}{{\bfseries Kodas}}&
		\multicolumn{1}{c|}{{\bfseries Reikalavimas}}&
		\multicolumn{1}{c|}{{\bfseries Svarba}}\\
		\hline
		\rowcolor{lightgray}
		\multicolumn{4}{|c|}{Prekės aukcionas}\\				
		\hline
		\multicolumn{2}{|c|}{FR 9.1}&
		{Puslapis „Prekės aukcionas“ turi būti matomas visiems prijungusiems naudotojams.
		}&		
		\multicolumn{1}{c|}{Būtina}\\
		\hline
		\multicolumn{1}{|c}{}&
		\multicolumn{1}{c|}{FR 9.2}&
		{Puslapyje „Prekės aukcionas“ aukcionas yra aktyvus tokį laikotarpį, kokį nurodė naudotojas sukūręs aukcioną.
		}&		
		\multicolumn{1}{c|}{Būtina}\\
		\hline	
		\multicolumn{1}{|c}{}&
		\multicolumn{1}{c|}{FR 9.3}&
		{Jei aukciono statusas yra „Aktyvus“, naudotojai gali siūlyti savo kainas, jas įvedus į kainos laukelį ir paspaudus mygtuką „Patvirtinti“.
		}&
		\multicolumn{1}{c|}{Būtina}\\									
		\hline
		\multicolumn{1}{|c}{}&
		\multicolumn{1}{c|}{FR 9.4}&
		{Naudotojo siūloma kaina negali būti mažesnė už jau pasiūlytą kainą. Jei naudotojas bando pasiūlyti mažesnę kainą, turi būti išmetamas klaidos pranešimas, jog kaina yra per maža.
		}&
		\multicolumn{1}{c|}{Būtina}\\									
		\hline
		\multicolumn{1}{|c}{}&
		\multicolumn{1}{c|}{FR 9.5}&
		{Praėjus naudotojo nurodytam aukciono laikotarpiui jo statusas turi automatiškai pasikeisti į „Pasibaigęs“, puslapyje „Prekės aukcionas“ turi būti parašytas aukciono laimėtojas,o kainos siūlymo laukelis nebeturi būti matomas.
		}&
		\multicolumn{1}{c|}{Būtina}\\									
		\hline
	\end{tabular}		
\end{table}

\subsection{Visi aukcionai}
\begin{table}[H]
	\caption{Funkciniai reikalavimai. Visi aukcionai}
	\begin{tabular}{|p{1cm}|p{1cm}|p{11,5cm}|p{3,5cm}|}
		\hline 
		\rowcolor{gray!50}
		\multicolumn{2}{|c|}{{\bfseries Kodas}}&
		\multicolumn{1}{c|}{{\bfseries Reikalavimas}}&
		\multicolumn{1}{c|}{{\bfseries Svarba}}\\
		\hline
		\rowcolor{lightgray}
		\multicolumn{4}{|c|}{Visi aukcionai}\\				
		\hline
		\multicolumn{2}{|c|}{FR 10.1}&
		{Puslapis „Visi aukcionai“ turi būti matomas tik sistemos administratoriams.
		}&		
		\multicolumn{1}{c|}{Būtina}\\
		\hline
		\multicolumn{1}{|c}{}&
		\multicolumn{1}{c|}{FR 10.2}&
		{Puslapyje „Visi aukcionai“ turi būti matomas visų aktyvių ir pasibaigusių aukcionų sąrašas.
		}&		
		\multicolumn{1}{c|}{Būtina}\\
		\hline	
		\multicolumn{1}{|c}{}&
		\multicolumn{1}{c|}{FR 10.3}&
		{Visas aukcionų sąrašas turi būti pateikiamas viename puslapyje.
		}&
		\multicolumn{1}{c|}{Būtina}\\									
		\hline
		\multicolumn{1}{|c}{}&
		\multicolumn{1}{c|}{FR 10.4}&
		{Puslapyje administratorius turi turėti galimybę ištrinti aukcioną.
		}&
		\multicolumn{1}{c|}{Būtina}\\									
		\hline
	\end{tabular}		
\end{table}

\subsection{Naudotojai}
\begin{table}[H]
	\caption{Funkciniai reikalavimai. Naudotojai}
	\begin{tabular}{|p{1cm}|p{1cm}|p{11,5cm}|p{3,5cm}|}
		\hline 
		\rowcolor{gray!50}
		\multicolumn{2}{|c|}{{\bfseries Kodas}}&
		\multicolumn{1}{c|}{{\bfseries Reikalavimas}}&
		\multicolumn{1}{c|}{{\bfseries Svarba}}\\
		\hline
		\rowcolor{lightgray}
		\multicolumn{4}{|c|}{Naudotojai}\\				
		\hline
		\multicolumn{2}{|c|}{FR 11.1}&
		{Puslapis „Naudotojai“ turi būti matomas tik sistemos administratoriams.
		}&		
		\multicolumn{1}{c|}{Būtina}\\
		\hline
		\multicolumn{1}{|c}{}&
		\multicolumn{1}{c|}{FR 11.2}&
		{Puslapyje „Naudotojai“ turi būti matomas visų naudotojų sąrašas.
		}&		
		\multicolumn{1}{c|}{Būtina}\\
		\hline	
		\multicolumn{1}{|c}{}&
		\multicolumn{1}{c|}{FR 11.3}&
		{Visas naudotojų sąrašas turi būti pateikiamas viename puslapyje.
		}&
		\multicolumn{1}{c|}{Būtina}\\									
		\hline
		\multicolumn{1}{|c}{}&
		\multicolumn{1}{c|}{FR 11.4}&
		{Puslapyje administratorius turi turėti galimybę užblokuoti naudotoją aukcioną.
		}&
		\multicolumn{1}{c|}{Būtina}\\									
		\hline
	\end{tabular}		
\end{table}

\section{NEFUNKCINIAI REIKALAVIMAI}
Andrius
\newpage
\section{STRUKTŪRINIS DALYKINĖS SRITIES MODELIS}
Kasparas
\newpage
\section{UŽDUOTYS}
Šioje dokumento dalyje yra pateikiamos sistemos atliekamos užduotys  (\ref{fig:usecase} pav.) Pateikiamas pagrindinis scenarijus ir alternatyvūs.\\
\begin{figure}[H]
\centering
\includegraphics[width=\linewidth]{img/UseCaseDiagram.png}
\label{fig:usecase}
\caption{Sistemoje atliekamos užduotys}
\end{figure}

\subsection{Naudotojo atliekamos užduotys}
Šiame skyrelyje išskiriamos visos sistemos naudotojo užduotys (\ref{fig:usercd} pav.), jas aprašant pateikiamas ne tik pagrindinis scenarijus, bet ir alternatyvūs. Taip pat atliekama robastiškumo analizė kiekvienai užduočiai.

\begin{figure}[H]
\centering
\includegraphics[width=\linewidth]{img/userusecased.png}
\label{fig:usercd}
\caption{Naudotojo atliekamos užduotys}
\end{figure}

\subsubsection{Užduotis - Registruotis}
Skyriuje pateikiamas užduoties registruotis aprašymas. Pateikiama užduoties robastiškumo analizės diagrama.\\
\textbf{Užduotis:}  Registruotis \\
\textbf{Scenarijus:} Naudotojas prisijungimo lange paspaudžia mygtuką registruotis. Sistema atidaro registracijos langą. Naudotojas suveda prisijungimo vardą, slaptažodį  \\
\textbf{Alternatyvūs scenarijai:} \\
\textbf{Nuoroda į reikalavimą: } 


\newpage

\sectionnonum{REZULTATAI}
\end{document}
